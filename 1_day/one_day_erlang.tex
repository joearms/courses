\documentclass[12pt]{article}

%% \usepackage{sslides}
%% \RequirePackage{fancyvrb} 

\usepackage{verbatim}
\usepackage{graphicx}
\usepackage[usenames]{color}
\definecolor{Red}{rgb}{1,0,0}
\definecolor{Blue}{rgb}{0,0,1}
\definecolor{Green}{rgb}{0,1,0}

\definecolor{Emph}{rgb}{0.7,0.1,0.1}  %softer red for display
\renewcommand{\emph}[1]{\textcolor{Emph}{\bf\textit{#1}}}

%% define color for slide titles
%\definecolor{Title}{rgb}{0,0,1}  %blue
%% \definecolor{Title}{rgb}{0.2,0.2,0.6}  %softer blue for display
%\definecolor{Title}{rgb}{1,0,0}  %red

%% the next three lines fix an error in the foramtting
%% of the table of contents

\makeatletter
\renewcommand\l@subsection{\@dottedtocline{2}{1.3em}{3em}}
\makeatother



%%----------------------------------------------------------------
\begin{document}
\title{An Introduction to Elang}

\author{Joe Armstrong $<$erlang@gmail.com$>$\\
(c) 2018 Joe Armstrong.\\
 Whoomph Software AB \\
  All rights Reserved.}


 \maketitle
 
 \tableofcontents

\section{To whet your appetite}

By the end of this course I hope you'll be able to {\sl read} Erlang
programs - and you'll have started to be able to {\sl write} programs.

To start you off I'm going to show you a small program that by the
end of the day you should be able to understand. I'll show it in
a moment, but before this we have to talk about patterns.

\subsection{Patterns}

Patterns are central to Erlang. We'll meet them in many contexts.

In expressions:

\verb+ Pattern = Value+

In function definitions:

\begin{verbatim}
funcName(Pattern1) -> Actions1;
funcName(Pattern2) -> Actions1;
...
funcName(PatternN) -> ActionsN.
\end{verbatim}

In \verb+case+ statements:

\begin{verbatim}
case Value of
   Pattern1 -> Actions1;
   Pattern2 -> Actions1;
   ...
   PatternN -> ActionsN
end
\end{verbatim}

In funs (or closures):

\begin{verbatim}
F = fun(Pattern1) -> Actions1;
       (Pattern2) -> Actions1;
       ...
       (PatternN) -> ActionsN
    end,
\end{verbatim}

In message reception:

\begin{verbatim}
receive
   Pattern1 -> Actions1;
   Pattern2 -> Actions1;
   ...
   PatternN -> ActionsN
end
\end{verbatim}

\subsection{The Multi Server}

\begin{verbatim}
-module(multi_server).
-compile(export_all).

start() -> spawn(fun() -> loop() end).

loop() ->
    receive
        {_Pid, {email, _From, _Subject, _Text} = Email} ->
            io:format("multi_server:email:~p~n",[Email]),
            {ok, S} = file:open("inbox", [write,append]),
            io:format(S, "~p.~n", [Email]),
            file:close(S);
        {_Pid, {im, From, Text}} ->
            io:format("Msg (~s): ~s~n",[From, Text]);
        {Pid, {get, File}} ->
            io:format("multi_server:get:~s~n",[File]),
            Pid ! {self(), file:read_file(File)};
        Any ->
            io:format("multi server got:~p~n",[Any])
    end,
    loop().
\end{verbatim}

This program is:

\begin{itemize}
\item An email server - (SMTP becomes an erlang message)
\item An instant messgaging agent (jabber?, XMPP)
\item A web server (or FTP server, or both) (HTTP, ...)
\end{itemize}

\section{Getting started}

\begin{itemize}
\item Starting the shell
\item Entering commands
\item Stopping the shell
\item Editing commands
\item Data types
\item Variables
\item Pattern matching
\end{itemize}

\subsection{Starting the shell}

\begin{itemize}
\item Unix-based systems:  \verb+$ erl+ 
\item Windows:             \verb+Programs -> OTP ... -> Erlang+ 
\end{itemize}

This is what you see

\begin{verbatim}
$ erl
Erlang (BEAM) emulator version 5.6 [source] [smp:4] [async-threads:0] 
             [hipe] [kernel-poll:false]

Eshell V5.6  (abort with ^G)
1> 
\end{verbatim}

\subsection{Shell commands}

\begin{itemize}
\item Shell is \verb+read-eval-print+ loop
\item Commands end with a period (.) followed by whitespace
\end{itemize}

\begin{verbatim}
1> 3 * 7.
21
2> math:sqrt(2).
1.41421
3> "hello". "world".
"hello"
4> "world".
"world"
\end{verbatim}

\subsection{Shell commands}

\begin{itemize}
\item Repeated prompt means command is not yet finished
\end{itemize}

\begin{verbatim}
5> 123 +
5> 3456 *
5> 789.
2726907
6>
\end{verbatim}

\subsection{Stopping the shell}

\begin{itemize}
\item (ctrl) + C
\end{itemize}

\begin{verbatim}
BREAK: (a)bort (c)ontinue (p)roc info (i)nfo (l)oaded
       (v)ersion (k)ill (D)b-tables (d)istribution
a
\end{verbatim}

\begin{itemize}
\item (ctrl) +\ -- immediate exit
\item init:stop(). -- controlled exit
\item erlang:halt() -- uncontrolled exit
\end{itemize}

\subsection{Recall and edit commands in the shell}

\begin{itemize}
\item Arrow keys
\item Emacs editor commands
\begin{itemize}
\item Ctrl + B -- back
\item Ctrl + F -- forward
\item Ctrl + P -- previous
\item Ctrl + N -- next
\end{itemize}
\item Tab -- try to expand module or function names
\end{itemize}

\subsection{Exercise}

Try it out. Fire up erlang at a command prompt. Do some basic arithmetic.

\begin{verbatim}
$ erl
1> 123456789 * 987654321 * 123456789.
15053411111487447638891241
...
\end{verbatim}

\subsection{The shell}

\begin{itemize}
\item Major caveat: you cannot define functions in the shell
\item Instead, create in separate files and compile in the shell
\end{itemize}

\section{Variables}

\begin{itemize}
\item Variables start with an upper case letter
\item syntax \verb+[A..Z][a..zA..Z0..9]+ (almost)
\end{itemize}

\begin{verbatim}
1> MinutesPerHour = 60.
60
2> HoursPerDay = 24.
24
3> MinutesPerDay = MinutesPerHour * HoursPerDay.
1440
4> 
\end{verbatim}

\subsection{Variables cannot vary}

\begin{verbatim}
4> Count = 1.
1
5> Count = Count + 1.
** exception error: no match of right hand side value 2
\end{verbatim}

\begin{itemize}
\item These are not like Java or C\# variables
\item More like variables in math
\end{itemize}

\subsection{Back to high school}

\begin{verbatim}
     x + 3y = 15
     x - 3y = 3

     x = ?, y = ?
\end{verbatim}


\verb+x+ and \verb+y+ mean the same thing in all equations.

\subsection{Single assignment variables}

\begin{itemize}
\item As in math, \verb+X = X + 1+ is illegal
\item Enables concurrency
\item No locks needed since we can't share and update data
\item Makes debugging easy (because variable is only set once)
\item Lives as long as it is within scope
\end{itemize}

\section{Pattern matching}

\begin{itemize}
\item \verb+Pattern = Value+ matchs Pattern against Value
\item \verb+X = 10+
means "match the pattern \verb+X+ against the value \verb+10+"
\end{itemize}

\subsection{Var = value}

\begin{itemize}
\item if \verb+Var+ is a "fresh" (new , unbound) variable, then the pattern is matched by giving \verb+Var+ the value \verb+Value+
\item if \verb+Var+ already has a value, pattern matches if its value equals \verb+Value+, fails otherwise
\end{itemize}

\begin{verbatim}
1> X = 10.
10
2> X = 10.  % matches
10
3> X = 2*4 + 2.  % matches
10
4> X = 11.
** exception error: no match of right hand side value 11
\end{verbatim}


\section{Data types}

Simple data types:

\begin{itemize}
\item Integers 
\begin{itemize}
\item \verb+1 45 123123 2#101012 8#0723 16#face+
\end{itemize}
\item Floats 
\begin{itemize}
\item \verb+12.34e-07 3.14159+
\end{itemize}
\item Atoms 
\begin{itemize}
\item \verb+true false hello_world 'funny atom #$%!!'+
\item atoms are like enumerated types in C
\item (\verb+monday+, \verb+tuesday+, \verb+wednesday+, ... are atoms -- these can be used
to represent days of the week)
\end{itemize}
\end{itemize}

\subsection{Complex types - Tuples}

\begin{itemize}
\item Ordered list of values
\verb+{food, egg}+
\item can be nested
\verb+{person,{firstname,"joe"},{lastname,"armstrong"}}+
\end{itemize}

\subsection{Creating and unpacking a tuple}

\begin{itemize}
\item Constructed using \verb+{+ ... \verb+}+ notation
\item Unpacked by pattern matching
\end{itemize}

\begin{verbatim}
4> Person = {person, "Joe", "Armstrong"}.
{person, "Joe", "Armstrong"}
5> {Type, First, Last} = Person.
{person, "Joe", "Armstrong"}
6> Type.
person
7> Last.
"Armstrong"
\end{verbatim}

\subsection{Repeated occurrence of variables}

If a variable occurs more than once in a pattern,
it must have the same value.

\begin{verbatim}
1> {X, Y, X} = {1, a, 1}.
{1,a,1}
2> X.
1
3> Y.
a
4> {A, B, A} = {1, a, 2}.
** exception error: no match of right hand side value {1,a,2}
\end{verbatim}


\subsection{Anonymous variables}

\verb+_+ is a "don't care" variable. Used as a placeholder in pattern
matches.

\begin{verbatim}
1> {X, _, X} = {1, a, 1}.
{1,a,1}
2> X.
\end{verbatim}

\subsection{Complex types - Lists}

\begin{itemize}
\item Sequence of values (can be different types)
\item \verb+[]+ is the empty list
\item \verb+[X1,X2,...,Xn]+ is a list with elements \verb+X1+, \verb+X2+, ..., \verb+Xn+
\item \verb+[...]+ in a pattern deconstructs the list
\end{itemize}

\begin{verbatim}
1> X = [1, 2, a, b].
[1,2,a,b]
2> [A, B, C, D] = X.
[1,2,a,b]
3> A.
1
4> D.
b
\end{verbatim}

\subsection{Lists as recursive structures}

\begin{itemize}
\item First element of a list is the {\sl head}, what remains is a list
called the {\sl tail}. (Lisp programmers, think CAR and CDR)

\item \verb+[ ]+ is the empty list
\item \verb+[ H | T ]+ is the list with head \verb+H+ and tail \verb+T+ (which must be a list)

\item Can be used to create and to pattern match
\end{itemize}

\begin{verbatim}
1> X = [1,2,3,4].
[1,2,3,4]
2> Y = [abc|X].
[abc,1,2,3,4]
3> [X1,_,X2|X4] = Y
[abc,1,2,3,4]
4> X3.
2
5> X4.
[3,4]
\end{verbatim}


\subsection{Strings}

\begin{itemize}
\item \verb+$a+ is short for \verb+97+ (The ASCII code for the character \verb+a+)
\item \verb+"abc"+ is shorthand for \verb+[$a, $b, $c]+
\item The shell prints lists of integers as strings (if it can)
\item \verb|++| concatenates lists
\end{itemize}


\begin{verbatim}
1> Name = "Armstrong".
"Armstrong"
2> "Joe " ++ Name.
"Joe Armstrong"
3> [ 99, 97, 110].
"can"
\end{verbatim}

\subsection{Quiz: If strings are lists...}

\begin{verbatim}
1> A = "Hi,".
"Hi,"
2> B = "there".
"there"
3> length([A, B]).
?
4> length(A ++ B).
?
5> length([A | B]).
?
\end{verbatim}

\subsection{Complex types - Maps}

\begin{verbatim}
1> Person = #{first => "Joe", last => "Armstrong"}.
#{first => "Joe",last => "Armstrong"}
2> maps:get(first, Person).                        
"Joe"
3> maps:get(age, Person).  
** exception error: {badkey,age}
     in function  maps:get/2
        called as maps:get(age,#{first => "Joe",last => "Armstrong"})

4> maps:find(age,Person).
error
5> maps:find(first,Person).
{ok,"Joe"}
\end{verbatim}

You can pattern match on maps:

\begin{verbatim}
birthday(#{age := N} = Person) ->
    Person#{arg => N+1}.
\end{verbatim}

\subsection{Exercise}
\begin{itemize}
\item Try to predict the result of each of the following {\sl before}
typing it into the shell. Use the command \verb+f().+ after each
command to {\sl forget} any bindings. \verb+b()+ prints all bindings.
\end{itemize}

\begin{verbatim}
X = true.
{X,abc} = {123,abc}.
{X,Y,Z} = {222,def,"cat"}.
{X,Y} = {333,ghi,"cat"}
{X,Y,X} = {{abc,12},42,{abc,12}}.
{X,Y,X} = {{abc,12},42,true}.
[H|T] = [1,2,3,4,5].
[H|T] = "cat".
[A,B,C|T] = [a,b,c,d,e,f].
\end{verbatim}


\subsection{Shell commands}
\begin{verbatim}
> help().
 b()         -- display all variable bindings        
 e(N)        -- repeat the expression in query <N>   
 f()         -- forget all variable bindings         
 f(X)        -- forget the binding of variable X     
 h()         -- history                              
 history(N)  -- set how many previous commands to keep 
 results(N)  -- set how many previous command results to keep
 v(N)        -- use the value of query <N>                   
\end{verbatim}

...  many more commands ...

\section{Compiling and testing}

\begin{itemize}
\item Step 1 - Write a unit test
\item Step 2 - Write some code
\item Step 3 - Iterate 1 and 2
\end{itemize}

(In most of the exercises I'll give you the unit tests.)

\subsection{Write the unit test}

\verbatiminput{code/compiling/math3a.erl}

But...

\begin{verbatim}
1> c(math3a).
./math3a.erl:5: function sum/1 undefined
error
\end{verbatim}

\subsection{Correct the code and run}

\verbatiminput{code/compiling/math3b.erl}

\begin{verbatim}
1> c(math3b).
{ok,math3b}
2> math3b:test().
** exception error: no function clause matching math3b:sum([ ])
     in function  math3b:sum/1
     in call from math3b:test/0
\end{verbatim}

\subsection{Try again}

\verbatiminput{code/compiling/math3c.erl}

\begin{verbatim}
7> c(math3c).     
{ok,math3c}
8> math3c:test().
horray
\end{verbatim}

\subsection{Anatomy of a function}

\verbatiminput{code/compiling/amod.erl}

\subsection{Useful Trick when developing}

\verb+-compile(export_all)+ exports all functions from this module.

\subsection{Running your program}

\begin{itemize}
\item Compile it \verb+$ erlc mycode+
\item Run it \verb+$ erl -noshell -pa /path/to/code -s Mod Func Args ...+
\end{itemize}

\subsection{Or use escript}

\begin{verbatim}
#!/opt/local/bin/escript 

main(List) ->
    io:format("Args are ~p~n", [List]).
\end{verbatim}

Then, from the command line

\begin{verbatim}
% chmod +x myprog
% ./myprog 34 67
Args are ["34","67"]
\end{verbatim}

\subsection{Code paths etc.}

\begin{itemize}
\item Erlang autoloads compiled modules using the current code search paths

\item For libraries, can set the path in \verb+./.erlang+ or \verb+${HOME}/.erlang+
\item both are read by \verb+erl+ when it starts

\item Can contain any \verb+erl+ commands 
\end{itemize}

\begin{verbatim}
code:add_patha("/Users/joe/Desktop/work/2007/jaerlang").
code:add_patha("/Users/joe/Desktop/work/2007/jaerlang/socket_dist").
\end{verbatim}

\subsection{Exercise:  During an idle moment...}

Try adding \verb+io:format("Hello from:~p~n",[file:get_cwd()]).+ 
to either \verb+/.erlang+ or \verb+${HOME}/.erlang+.

\section{Getting help}

Manual pages are not installed by default (but MacPorts does load 
them for you)

\begin{itemize}
\item Unix --
  \verb+wget http://www.erlang.org/download/otp_doc_man_R11B-5.tar.gz+
  \begin{itemize}
  \item Unpack so that the \verb+man+ directory is unpacked at the root
    of the Erlang distribution (usually \verb+/usr/local/lib/erlang+).
  \item View using \verb+erl -man Modulename+
  \end{itemize}
    
\item Windows -- the HTML rendered documentation is included in the distribution
\end{itemize}

\section{Sequential erlang}

\subsection{Review: Types}

Sequential Erlang programs manipulate instances of data types

\begin{itemize}
\item \verb+integers+ -- \verb+2176537165+ \verb+1+ \verb+16#face+
\item \verb+floats+ -- \verb+1.24234+
\item \verb+atoms+ -- \verb+monday+, \verb+tuesday+
\item \verb+binaries+ -- \verb+<<"3868GAJSJB">>+ memory buffers
\end{itemize}

With two complex data types

\begin{itemize}
\item \verb+lists+ -- \verb+[X1, X2, ...]+
\item \verb+tuples+ -- \verb+{X1, X2, ...}+
\end{itemize}

\subsection{Functions}

Remember:

\begin{verbatim}
func(Pattern1) -> Actions1;
func(Pattern2) -> Actions2;
...
func(PatternN) -> ActionsN.
\end{verbatim}

{\bf Get the punctuation right - SEMI-COLON ... DOT.}

\begin{verbatim}
temp_convert({f, F}) -> {c, 5*(F-32)/9};
temp_convert({c, C}) -> {f, 32 + 9*C/5}.
\end{verbatim}

\begin{verbatim}
lookup(Key, [{Key,Val}|_]) -> {yes, Val};
lookup(Key, [_|T]) -> lookup(Key, T);
lookup(Key, []) -> no.
\end{verbatim}

\subsection{Built-In functions (BIFs)}

\begin{itemize}
\item We can't do everthing with pattern matching For example, type
  conversion BIFS, convert an atom to a list, convert a tuple to a
  list etc.  
\item \verb+atom_to_list(Atom)+, \verb+tuple_to_list(Tuple)+,
  \verb+term_to_binary(term)+, \verb+binary_to_term(Bin)+
\item \verb+erl -man erlang+ describes all BIFS
(book pages 451--460 contains a summary)
\end{itemize}

\subsection{BIFs For type conversion}

\begin{verbatim}
gen_tcp:send(Socket, term_to_binary(Term)), ...

receive
   {tcp_data, Socket, Bin} ->
        Term = binary_to_term(Bin)
   ...
end.
\end{verbatim}

\subsection{BIFS for efficiency}

\begin{verbatim}
encode(Term) ->
   B = term_to_binary(Term),
   Md5 = erlang:md5(B),
   <<B/binary,Md5/binary>>.

decode(<<Md5:16/binary,B/binary>>) ->
   Md5 = erlang:md5(B), %% throws an error if wrong value
   binary_to_term(B).
\end{verbatim}

\subsection{BIFS}

\begin{itemize}
\item Quirky syntax
\item Sometimes \verb+tuple_to_list+, sometimes \verb+erlang:now()+
\item In theory the \verb+erlang:+ BIFS are implement in Erlang (not true :-)

\item Behave as if they were defined in the module \verb+erlang+
\end{itemize}

\subsection{apply}

You can build a call to a function and apply that function at run-time
using \verb+apply(Mod, Func, [Arg1, Arg2, ...])+.

\begin{verbatim}
1> M = list_to_atom("erlang").       
erlang
2> F = list_to_atom("tuple_to_list").
tuple_to_list
3> apply(M, F, [{a,b,c}]).           
[a,b,c]
4> M:F({a,b,c}).
[a,b,c]
\end{verbatim}

\subsection{Funs}

Funs are ``anonymous functions'' (also called lambda-expressions, closures).

\begin{verbatim}
1 > Double = fun(X) -> 2*X end.
#Fun<erl_eval.4.4564646>
2 > Double(2).
4
\end{verbatim}

Funs can be arguments to functions.

\begin{verbatim}
3 > lists:map(Double, [1,2,3,4]).
[2,4,6,8]
\end{verbatim}

\subsection{Let's add a for loop to Erlang}

\verbatiminput{code/sequential/hofs.erl}

\begin{verbatim}
1> c(hofs).
{ok,hofs}
2> hofs:for(1,10,fun(I) -> 2*I end).
[2,4,6,8,10,12,14,16,18,20]
\end{verbatim}

\subsection{Funs can return funs}

\begin{verbatim}
1> MakeAdder = fun(Inc) -> fun(X) -> X + Inc end end.
#Fun<erl_eval.6.72228031>
2> Add5 = MakeAdder(5).
#Fun<erl_eval.6.72228031>
3> Add5(10).
15
\end{verbatim}

\subsection{map}

\begin{itemize}
\item apply the same function to every element of a list.
The result is a list the same length as the input list
\item \verb+map(F, [X1,X2,...Xn]) -> [F(X1), F(X2), ..., F(Xn)]+
\item {\sl Incredibly useful}
\end{itemize}

\begin{verbatim}
 1 > lists:map(fun(X) -> X * X end, [1,2,3]).
 [1,4,9]
\end{verbatim}

\subsection{Writing our own map}


\begin{verbatim}
my_double([ H | T ]) -> [ 2*H | my_double(T) ];
my_double([]) -> [].
\end{verbatim}

\subsection{Writing our own map}

\begin{verbatim}
my_double([ H | T ]) -> [ 2*H | my_double(T) ];
my_double([]) -> [].
\end{verbatim}

Generalize:

\begin{verbatim}
my_map(Fun, [ H | T ]) -> [ Fun(H) | my_map(Fun, T) ];
my_map(Fun, [])  -> [].

my_double(L) -> my_map(fun(X) -> 2*X end, L).
\end{verbatim}

\subsection{Accumulators}

Let's write \verb+sum+ using an accumulator:

\begin{verbatim}
sum(L) -> sum(L, 0).

sum([H|T], Sum) -> 
    Sum1 = Sum + H,
    sum(T, Sum1);
sum([], Sum) ->
    Sum.
\end{verbatim}

Note we have two different functions, \verb+sum/1+ and
\verb+sum2/2+. This is a common pattern--the additional parameters are
analogous to function-local variables in other languages.

\subsection{List comprehensions}


\verb+[ Constructor || Pattern <- List, Predicate, ...]+ 


\begin{itemize}
\item Come from set theory--the set of all \verb+X+ such that \verb+Y+, 
eg \verb+{ x : x is even }+
\item In Erlang, use \verb+||+
\end{itemize}

\begin{verbatim}
2> L = lists:seq(1,20).
[1,2,3,4,5,6,7,8,9,10,11,12,13,14,15,16,17,18,19,20]
3> [ X || X <- L, (X rem 2) =:= 0].
[2,4,6,8,10,12,14,16,18,20]
\end{verbatim}

* Allows very compact code

\verb+map(F, L) -> [F(X) || X <- L].+

\subsection{Quicksort}

\verbatiminput{code/sequential/qsort.erl}

\begin{verbatim}
1> c(qsort).                  
{ok,qsort}
2> qsort:sort([1,a,3,4,b,c,2]). 
[1,2,3,4,a,b,c]
\end{verbatim}

\subsection{Multiple return values}

\begin{itemize}
\item Single return value \verb+X = some_func(....)+
\item To get multiple return values use a tuple\\
  \verb+{X, Y, Z} = another_func(....)+ 
\end{itemize}

\subsection{Accumulators}
\begin{itemize}
\item Pass extra argument(s) into the function
\end{itemize}

\verbatiminput{code/sequential/accum.erl}

\begin{verbatim}
1> accum:evens_and_odds([1,2,3,4,5,6,7]).
{[6,4,2],[7,5,3,1]}
\end{verbatim}

\subsection{Accumulators}
\begin{itemize}
\item User \verb+reverse+ to fix order at end
\end{itemize}

\verbatiminput{code/sequential/accum2.erl}

\begin{verbatim}
2> accum:evens_and_odds([1,2,3,4,5,6,7]).
{[2,4,6],[1,3,5,7]}
\end{verbatim}

\subsection{Guards}

Extend function pattern matching to include additional predicates and
comparisons

\begin{verbatim}
func(A, B)  when  ...guard...  -> 
\end{verbatim}

For example:

\begin{verbatim}
func1(A, B) when is_integer(A) ->
    ...
\end{verbatim}

\subsection{Single Guard conditions}
\begin{itemize}
\item Type-match predicates (\verb+is_integer(X)+, \verb+is_tuple(X)+, ...)
\item Built-in guard functions (\verb+length(X)+, \verb+hd(X)+, ...)
\item comparisons, boolean, and arithmetic expressions
\end{itemize}

\begin{verbatim}
func1(A, B) when is_integer(A) andalso A + 1 > B ->
    ...
\end{verbatim}

\subsection{Full Guard}

\begin{itemize}
\item A comma separated list of single guard conditions
\item Matches only if all match
\end{itemize}

\begin{verbatim}
func1(A, B) when is_integer(A), is_tuple(B) ->
    ...
\end{verbatim}

\subsection{Guard Sequence}

\begin{itemize}
\item A semicolon separated list of full guards
\item matches if any match
\end{itemize}

\begin{verbatim}
func1(A, B) when is_integer(A), is_tuple(B); is_float(A), is_tuple(B) ->
    ...
\end{verbatim}

Or

\begin{verbatim}
classify(Day) when A =:= saturday; A =:= sunday ->
    weekend;
classify(Day) ->
    weekday.
\end{verbatim}

\subsection{case and if}

\begin{verbatim}
goto_work(Day) ->                   goto_work(Day) ->
    case classify(Day) of              if
        weekday -> true;                  Day =:= sunday -> false;
        weekend -> false                  Day =:= saturday -> false;
    end.                                  true             -> true
                                       end.
\end{verbatim}

Remember:

\begin{verbatim}
case Value of
    Pattern1 -> Actions1;
    Pattern2 -> Actions2;
    ...
    PatternN -> ActionsN
end
\end{verbatim}

{\bf Get punctuation right SEMI-COLONS.}


\subsection{Booleans}

\begin{itemize}
\item \verb+true+ and \verb+false+ atoms (not built-in type)
\item By convention only, but a strongly recommended convention.
\end{itemize}


\subsection{Many functions work with booleans}

\begin{itemize}
\item \verb+lists:filter(Fun, L) -> L'+


This takes a list of values \verb+[X1,X2,...,Xn]+ and produces a new
list \verb+[Xi,Xj,...]+ of all the vlaues in \verb+L+ for which \verb+Fun(X)+ is
\verb+true+.

\verb+lists:partition(Fun, L) -> {T, F}+

Splits the elements \verb+X+ in \verb+L+ into two sub-lists \verb+T+
and \verb+F+ depending on whether \verb+Fun(X)+ is \verb+true+ or
\verb+false+.
\end{itemize}

These functions will not work on things like \verb+on+ and \verb+off+.

\subsection{Tame large tuples with records}

\verb+{person,"joe","armstrong"}+ is fine when there are just a few
elements but when we have more than (say) half a dozen elements things
get messy.

\begin{verbatim}
-record(person, {name,firstname,lastname,age,sex,weight,height}).

X = #person{name="jane", age=32, sex=female} %% creates a new record

birthday(#person{age=N} = X) ->
    X#person{age = N+1}
\end{verbatim}

Records both construct and pattern match.

\subsection{Record defaults}

\begin{verbatim}
-record(window, {width=100, ht=100, x=10, y=10, color=red}).

X = #window{width=200}, %% other arguments get default values
\end{verbatim}

\subsection{Not mentioned}

\begin{itemize}
\item macros

\item parse transforms
\end{itemize}

\subsection{Exercises: }

Write a program to number the elements in a list.
Here's the test case:

\begin{verbatim}
test() ->
  [{1,a},{2,b},{3,c}] = number_list([a,b,c]),
\end{verbatim}


Write functions \verb+one(List)+, \verb+two(List)+, and
\verb+three(List)+ that each returns two lists containing the even and
odd elements in \verb+List+. Use three different techniques.  The test
case is:

\begin{verbatim}
test() ->
    {[2,4],[1,3,5]} = one([1,2,3,4,5]),
    {[2,4],[1,3,5]} = two([1,2,3,4,5]),
    {[2,4],[1,3,5]} = three([1,2,3,4,5]),
    hooray.
\end{verbatim}

\section{Exceptions}

\begin{itemize}
\item Type errors occur when calling a BIF with the wrong type
\item Pattern matching errors
\item Explicit errors caused by program executing \verb+throw+,
  \verb+exit+, \verb+error+
\item exceptions can be {\sl converted} into values with
\verb+catch+  and\\
\verb+try ... catch ... after ... end+
\end{itemize}

\subsection{Convert exceptions into values}

\begin{verbatim}
1> X = atom_to_list(123).
** exited: {badarg,[{erlang,atom_to_list,[1234]},
                    {erl_eval,do_apply,5},
                    {erl_eval,expr,5},
                    {shell,exprs,6},
                    {shell,eval_loop,3}]} **
2> X.
** 1: variable 'X' is unbound **
\end{verbatim}

\begin{verbatim}
3> X = (catch atom_to_list(123)).
{'EXIT',{badarg,
  [{erlang,atom_to_list,[1234]},
   {erl_eval,do_apply,5},
   {erl_eval,expr,5},
   {erl_eval,expr,5},
   {shell,exprs,6},
   {shell,eval_loop,3}]}}
4> X.
{'EXIT',{badarg,
  [{erlang,atom_to_list,[1234]},
   {erl_eval,do_apply,5},
   {erl_eval,expr,5},
   {erl_eval,expr,5},
   {shell,exprs,6},
   {shell,eval_loop,3}]}}
\end{verbatim}

\subsection{Protecting a function call}

\verb+catch F(X)+ returns \verb+F(X)+ if no exception is generated
while evaluating the function

\begin{verbatim}
some_function(X, Y) ->
    case (catch something_that_might_crash(X, Y)) of
        {'EXIT', Why} -> 
            %% .. handle the error ...;
        NormalReturn ->
            %% regular processing
    end.
\end{verbatim}

Evaluating \verb+some_function(X, Y)+ will never raise an exception
(assuming that we correctly handle the error in the \verb+EXIT+ branch
of the \verb+case+ statement)

\subsection{Protecting a sequence of expressions}

\begin{verbatim}
start(Arg1, Arg2) ->
    try 
       begin
           ok = start_server(),
           L1 = dothis(Arg1),
           ...
       end
    catch
       exit: Pattern ->
          ...
    after
        ....
    end
\end{verbatim}

\subsection{Use exceptions for "impossible" situations}

I once found this code:

\begin{verbatim}
opcode(load) -> 1;
opcode(store) -> 2;
opcode(X) ->    %% invalid: what should I do now?
    io:format("bad opcode:~p~n",[X]).
\end{verbatim}

What does this return for invalid opcodes?

\subsection{Solutions}

{\sl do something}

\begin{verbatim}
opcode(load) -> 1;
opcode(store) -> 2;
opcode(X) -> exit({ebadOpCode, X}).
\end{verbatim}

Solution two: {\sl do nothing, be lazy} (best)

\begin{verbatim}
opcode(load) -> 1;
opcode(store) -> 2.
\end{verbatim}

Program will generate an exception when called with an invalid
argument {\sl precisely as in solution two, the only difference
is the name of the exception}.

\subsection{exit, throw, or error}

\begin{itemize}
\item use \verb+exit+ if this is so serious that the process should die
\item use \verb+thow+ if you intend to catch the error in the program
\item use \verb+error+ if you want to behave like a "normal" library error
\end{itemize}

\subsection{Defensive programming is painful...}

\begin{verbatim}
case file:open("config") of
    {ok, T} ->
        case T of
           {port,P,host,H} ->
              case open_socket(Host, Port) of
                  {ok, Pid} ->
                     ...
                  {error, Why} ->
                     ...
              end
           ...  ->
              ...
         end
      ...
\end{verbatim}

\subsection{So don't program defensively}

\begin{itemize}
\item Use {\bf ONE BIG CATCH AT THE TOP OF YOUR PROGRAM}
\item write {\sl everything else} as if it is going to work.
\item let your code crash early
\end{itemize}

\begin{verbatim}
start(...) ->
    try
        begin
            {ok, T} = file:consult("config"),
            {port,Port,host,Host} = T,
            {ok, Pid} = open_socket(Host, Port)
        end;
    catch
      What:Why ->
          io:format("oooo err: ~p ~p~n",[What,Why])
    end
\end{verbatim}

(We'll talk about inter-process errors shortly...)

\subsection{Exercise:}

The BIF \verb+list_to_integer("1234")+ will return \verb+1234+, but
exits with a bad argument exception if given a bad string. Write a
function \verb+list_to_int(String)+ that returns \verb+{ok, Int)+ or
  \verb+{error, eBadInt}+.

The unit test is:

\begin{verbatim}
test() ->
    {ok, 123} = list_to_int("123"),
    {error, eBadInt} = list_to_int("abc"),
    hooray.
\end{verbatim}

\subsection{Debugging}

\begin{itemize}
\item There is a debugger--very few people use it
\item Errors are pretty easy to find. This is a
consequence of single assignment
\item Compiler errors are "obvious"
\item \verb+lib_misc:dump(File, Term)+ for large data structures
\item \verb+io:format("...~p....~n", [Var1, ...])+
\item \verb+?NYI+ macro
\item \verb+ifdef(DEBUGGING)+
\end{itemize}

\subsection{lib\_misc:dump(File, Term)}


\begin{verbatim} 
1 > V = epp:parse_file("ex1_bit_syntax.erl","",""),true.
true
2> lib_misc:dump("debug", V).
Dumping term to debug
ok
\end{verbatim}


\verb+V+ is stored in a file called \verb+debug+

\begin{verbatim}
{ok,
 [{attribute,1,file,{"ex1_bit_syntax.erl",1}},
  {attribute,1,module,ex1_bit_syntax},
  {attribute,2,compile,export_all},
  {function,
   31,
   test,
   0,
   [{clause,
     31,
     [],
     [],
     [{match,
       32,
       {var,32,'Code'},
    ...
\end{verbatim}

\subsection{MACRO: Not Yet Implemented}

\begin{verbatim}
-define(NYI(X),(begin 
                    io:format("*** NYI ~p ~p ~p~n",[?MODULE, ?LINE, X]),
                    exit(nyi) 
                end)).
\end{verbatim}

Use like this:

\begin{verbatim}
    ...
    ?NYI({glurk, X, Y}).
    ...
\end{verbatim}

\subsection{MACRO: DEBUGGING}

\begin{verbatim}
-define(DEBUGGING, true).
-ifdef(DEBUGGING).
-define(DEBUG(X), X).
-else.
-define(DEBUG(X), void).
-endif.

...
    ?DEBUG(io:format("Found:~p~n",[Files])),
...
\end{verbatim}
 
\section{Concurrency}

\begin{itemize}        
\item The world {\sl is} concurrent
\item Programming concurrent activities in a sequential language is
{\sl artificially difficult}
\item Modern processors are concurrent
\end{itemize}


\subsection{Models of concurrency}

\begin{itemize}

\item Shared memory (locks, mutexes, coroutines, processes,
threads, deadlock, livelock, failure, thread-safe)

\item Message passing 
\end{itemize}

\subsection{Processes or threads?}
 
\begin{itemize}
\item Threads {\sl share} resources (essentially an efficiency hack -- beware
of premature optimizations)
\item Processes do not share things. Processes are the basis of {\sl security} in an operating system

\item (Caution) the world ``process'' is thought of as a synonym for
  ``slow, big, difficult to manage'' Erlang processes are {\sl very
  lightweight}. (Much lighter than a conventional thread)
\end{itemize}

\subsection{Concurrency in other languages}

\begin{itemize}
\item Can only create very small number of processes (a few thousand at most)
\item Heavy weight
\item Only supports message passing and not the kind of error handling semantics that
Erlang has
\end{itemize}

\subsection{Basic ideas}

\begin{itemize}
\item If you know the name of a process, you can send it a message
\item \verb+Pid ! Message+
\item If \verb+Pid+ is hidden you cannot send a message to the
  processes (security)
\item Message arrives in a mailbox (like email)
\item You never know if a message arrives (if you want to be sure send a message back)
\item You can {\sl link} to a process. If a process dies and you linked
to it, you will be sent an error signal.
\item The error signal is like a message (more later)
\end{itemize}

\subsection{Everything is a process}

In the Erlang world we model all non-Erlang things as processes.

We interact with all objects by sending them messages. The behavior
of all objects in the external world is inferred from analysing the
messages they send to us. 

This is the {\sl purest} form of OO (encapsulation, isolation, polymorphic).

\subsection{Processes}

\begin{itemize}
\item \verb+spawn+ -- create a new process
\item \verb+send+ -- send a message to a process
\item \verb+receive+ -- receive a message
\end{itemize}

\subsection{Spawn}

Three syntaxes:
\begin{itemize}
\item \verb+spawn(fun foo/0)+ -- runs \verb+foo/0+ in a new process
\item \verb+spawn(fun() -> ... end)+ -- spawns an inline fun 
\item \verb+spawn(Mod, Func, [Arg1, Arg2, ..., Argn])+ performs\\
\verb+apply(Mod, Fun, [Arg1, Arg2, ..., Argn])+ in
a new process.
\end{itemize}

In \verb+Pid1+ evaluate \verb+Pid2 = spawn(fun() -> ...end)+

\includegraphics[width=10cm]{images/proc2.jpg}

\subsection{Send}

\verb+Pid ! Msg+ sends the message \verb+Msg+ to the 
to the {\sl Mailbox} of a process named \verb+Pid+.
\verb+Pid+ is the return value of of a previously evaluated \verb+spawn+
expression.

\subsection{Receive}

Receive suspends the process until a message arrrives in the mailbox
that matches one of the patterns in a \verb+receive+ statement.
(Rather like a fancy select statement)

Syntax:

\begin{verbatim}
receive
    Pattern1 -> Actions1;
    Pattern2 -> Actions2;
    ...
end
\end{verbatim}

\subsection{Including self()  in a message}

\begin{itemize}
\item \verb+self()+ is the pid of the current process
\item including \verb+self()+ in a message allows the recipient to respond

\includegraphics[width=10cm]{images/msgs1.jpg}

\subsection{Spawn send and receive example}

\verbatiminput{code/processes/counter1.erl}

\begin{verbatim}
1> c(counter1).     
{ok,counter1}
2> counter1:start().
Counter is now 1
bump
Got bumped
Counter is now 2
Got bumped
Counter is now 3
\end{verbatim}

\subsection{Selected or ordered receives}

\includegraphics[width=10cm]{images/msgs2.jpg}

Receive message from A or B whichever comes first

\includegraphics[width=10cm]{images/msgs3.jpg}

Receive message from A then B

\subsection{Sending Pids in messages}

\includegraphics[width=10cm]{images/msgs4.jpg}

\subsection{Counter processes}

How do we get the value of the counter from the process?

\verbatiminput{code/processes/counter2.erl}

\begin{verbatim}
1> c(counter2).
{ok,counter2}
2> P = counter2:start().
<0.38.0>
3> counter2:tick(P).
bump
4> counter2:tick(P).
bump
5> counter2:read(P).
3
\end{verbatim}

Is anything wrong with this code?

\subsection{Another process might reply}

``Better'' version -- include a return value to match on
that makes sure the expected process answers.

\verbatiminput{code/processes/counter3.erl}

\subsection{Exercise}

 \verb+multi_server+

 Write \verb+multi_server.erl+. \verb+Pid = multi_server:start()+
 starts a multi server.  It responds to three messages:

\begin{itemize}
\item \verb+{Pid, {email, Who, Subject, Text}}+ - a request to store email in your inbox
\item \verb+{Pid, {im, Who, Text}}+ - an instant message - write message to terminal.
\item \verb+{Pid, {get, File}}+ - a request to fetch a file. The server should reply
by sending the file contents (as a message) back to \verb+Pid+.

In all of these \verb+Pid+ is the process identifier of the process sending
a request to the server.
\end{itemize}

Hints:

\begin{itemize}
\item \verb+{ok, Bin} = file:read_file(File)+ reads \verb+File+ into a binary.
\item \verb+{ok, Stream} = file:open("inbox", [write,append])+ opens a file for write append
\item \verb+io:format(Stream, "~p.~n",[Term])+ writes a term to a file
\item \verb+file:close(Stream)+ closes a file
\end{itemize}


\subsection{Data abstraction (1)}

Did we need to "invent" the functions \verb+tick+ and \verb+read+?

\begin{verbatim}
2> c(counter4).
{ok,counter4}
3> Pid = spawn(fun() -> counter4:counter(0) end).
<0.42.0>
4> Pid ! bump.
bump
5> Pid ! bump.
bump
6> counter4:rpc(Pid, read).
2
\end{verbatim}

\verbatiminput{code/processes/counter4.erl}

\subsection{Data abstraction (2)}

\begin{itemize}
\item Are we hiding the wrong thing? Should we expose the protocol?
\item PROS:
Hides the messaging structure.
Makes "module resuability" easy.
We know how to define APIs.
\item CONS:
More code.
We don't know how to describe protocols.
Protocols do not exist in Erlang (or in any other language
for that matter) as first class objects.
\end{itemize}

\subsection{Abstracting Server Functionality }

\verbatiminput{code/processes/abs_server.erl}

\verbatiminput{code/processes/counter5.erl}

\begin{verbatim}
> Pid1 = counter5:start().
<0.36.0>
> counter5:tick(Pid1).
{cast,#Fun<counter5.0.20406117>}
> counter5:tick(Pid1).
{cast,#Fun<counter5.0.20406117>}
> counter5:read(Pid1)
2
\end{verbatim}

\subsection{Registered processes.}

Any program that wants to send a message to \verb+Pid+ must have
\verb+Pid+ in a local variable. This must be communicated to all
functions that want to send a messages to \verb+Pid+

This is secure (but inconvenient)

\begin{itemize}
\item \verb+register+ -- registers a global name with a \verb+Pid+
\item \verb+unregister+
* After \verb+register(name, Pid)+, \verb+name ! Term+ sends a message to \verb+Pid+
\item \verb+whereis(Name)+ returns \verb+Pid+ or \verb+undefined+
\end{itemize}

\subsection{Client-Server Model}

\verbatiminput{code/processes/reg_counter1.erl}

\subsection{Warnings}

If two processes do \verb+register+ with the {\sl same name at the
  same time} one will succeed and one will fail.

Suppose we evaluate the following in two parallel processes:

\begin{verbatim}
make_global(Name, F) ->
    register(Name, spawn(F)).
\end{verbatim}

One will fail the other will succeed. 

\begin{itemize}
\item Can you write a process \verb+make_global(Name, Fun)+ that is
  correct using only \verb+register+ and \verb+whereis+?

\item Think about it
\end{itemize}

\subsection{Timeouts:  receive ... after}

\begin{verbatim}
receive
    Pattern1 -> ...
    Pattern2 -> ...
after
    TimeInMilliseconds ->
        Actions
end.
\end{verbatim}

\includegraphics[width=10cm]{images/tick.jpg}

\subsection{Exercise: Write an alarm process}

\begin{itemize}
\item \verb+alarm(Pid, Msg, Time)+ sends the message \verb+Msg+ to
  \verb+Pid+ after \verb+Time+ milliseconds.
\item Use it to write "Hello, World!" to your console every two
  seconds
\end{itemize}


\subsection{Mutually recursive processes}

Process one needs to know \verb+Pid2+ and processes two needs to know
\verb+Pid1+

\begin{verbatim}
Pid1 = spawn(fun() -> start(F1) end),
Pid2 = spawn(fun() -> start(F2) end),
Pid2 ! Pid1,
Pid1 ! Pid2,
...
start(F) ->
         receive
             Pid -> F(Pid)
         end
\end{verbatim}

\subsection{Spawn process, send the code later}

\begin{verbatim}
Pid = spawn(fun() -> wait() end),
...
Pid ! {do, F}
...

wait() ->
    receive
       {do, F} ->
           F()
    end
\end{verbatim}

\subsection{SMP and processes}

\begin{itemize}
\item Erlang can take advantage of multiple processors and cores
\item Enable using \verb+-smp +s n + options (only if compiled in)
\end{itemize}

\subsection{Summary}

All concurrent behavior is programmed with

\begin{itemize}
\item \verb+spawn+
\item \verb+send+
\item \verb+receive+
\end{itemize}

Instead of {\sl blocks}, {\sl mutexes}, {\sl threads}, {\sl
  processes}, {\sl synchronised methods}, ...

\subsection{Client-Server}

{\bf The Most Important Concurrency Pattern}

\includegraphics[width=10cm]{images/cs.jpg}

\begin{verbatim}
-module(s1).
-export([start/0, f1/2, f2/3, ..., stop/0]).

start() -> register(s1, spawn(fun() -> loop(State1))).

%% interface routines
f1(A, B) -> rpc(s1, {do_f1,A,B}).
f2(...)  -> rpc(s1, ...).

rpc(Q) ->
    s1 ! {self(), Q},
    receive
        {s1, Reply} -> Reply
    end.
    
loop(State) ->
     receive
         {From, {do_f1, A, B}} ->
              ...
              FRom ! {s1, Reply},
              loop(State1);
         ...
     end.
\end{verbatim}

\subsection{Variations: Exit client on server error}

Exit the client if the RPC causes an error in the server

\begin{verbatim}
loop(State) ->
    receive
      ..
      {From, Q} ->    
          try  
             {Reply, State1} = f(Q, State),
             From ! {Name, ok, Reply},
             loop(State1);
          catch 
             exit:Why ->
                From ! {Name, die, Why},
                loop(State)
          end
\end{verbatim}

\begin{verbatim}
rpc(Name, Q) ->
    Name ! {self(), Q},
    receive
       {Name, ok, Reply} -> Reply;
       {Name, die, Why} -> exit(Why)
    end.
\end{verbatim}

\subsection{Variations:  timeout in client}

\begin{verbatim}
rpc(Name, Q) ->
    Name ! {self(), Q},
    receive
       {Name, ok, Reply} -> Reply;
       {Name, die, Why} -> exit(Why)
    after
       Time ->
           exit(timeout)
    end.
...
\end{verbatim}

\subsection{Variations:  delegation}

\begin{verbatim}
loop() ->
   receive 
     {From, Q} ->    
        ...
        Other ! {do, Q, replyAs, Name, replyTo, From}
\end{verbatim}
Delegated

\begin{verbatim}
loop() ->
   receive 
     {do, Q, replyAs, Name, replyTo, From} ->    
        ...
        From ! {Name, Reply}
\end{verbatim}
Responder

\begin{verbatim}
loop() ->
   receive 
     {From, Q} ->    
        ...
        From ! {Name, Reply}
\end{verbatim}
Original

\subsection{Variations: Mobile code}

\begin{verbatim}
loop(State) ->
   ... 
   {From, Func} ->    
        {Reply, State1} = Func(State),
        From ! {self(), Reply},
        loop(State1);
   ...
\end{verbatim}

\subsection{Variations: Mobile code with transactions}

\begin{verbatim}
loop(State) ->
   ... 
   {From, Func} ->  
        try Func(State) of
            {Reply, State1} ->
                From ! {self(), Reply},
                loop(State1);
        catch
           _:_ ->
                exit(From, evilLaugh),
                loop(State)
        end
   ...
\end{verbatim}

\subsection{Rolling your own}

\begin{itemize}
\item As you can see there are many variations of the client-server theme.
\item No two ways of writing the client-server loop do exactly the same thing.
These programs can be made to do exactly what you want. 
\item If you understand this then you can easily design your own middleware system
\end{itemize}

\section{Links}

\begin{itemize}
\item \verb+link+ -- link to a process
\item \verb+unlink+ -- remove the link
\item \verb+process_flag(trap_exit, true)+ -- receive a signal as a message
\end{itemize}

\subsection{What is a link?}


\includegraphics[width=10cm]{images/crash.jpg}

\begin{itemize}
\item \verb+links+ define error propagation paths
\item If \verb+A+ is linked to \verb+B+ then:
{\sl If \verb+A+ fails then \verb+B+ will be notified}
{\sl If \verb+B+ fails then \verb+A+ will be notified}
\item links are not "stacked" - calling \verb+link+ ten times has the same effect as
calling it once.
\item links propagate at the speed of messages - they are not instantaneous
(important in distributed systems)
\end{itemize}


\subsection{Messages and signals}

\begin{itemize}
\item Messages are sent with \verb+send+
\item Signals are sent when processes die. Signals are not messages
\item If \verb+A+ is linked to \verb+B, C, ...+ then \verb+B,C,...+ will be sent
an exit signal if \verb+A+ dies  
\item All processes die when they receive signals, unless they are
set to trap exits
\item To trap exits a process evaluates \verb+process_flag(trap_exit, true)+
\end{itemize}

\includegraphics[width=10cm]{images/trap.jpg}

\subsection{Exit messages}

\item \verb+{'EXIT', Pid, Why}+ can be received if you are trapping exits
\item If you are {\sl not} trapping exits and \verb+Why+ is not the
  atom \verb+normal+ you die
\end{itemize}


\subsection{Example (1) - monitor }

\verb+monitor(Pid)+ observer \verb+Pid+ and print a message if \verb+Pid+ dies.

\includegraphics[width=10cm]{images/trap.jpg}

\begin{verbatim}
monitor(Pid) ->
    link(Pid),
    process_flag(trap_exit, true),
    receive
        {'EXIT',Pid,Why} ->
            io:format("Process died Reason:~p~n",[Pid,Why])
    end.
\end{verbatim}

\subsection{The small print}

\begin{itemize}
\item You can fake an exit message by calling \verb+exit(Pid1, Why)+ -- If you are \verb+Pid2+ then \verb+Pid1+ sees \verb+{'EXIT',Pid2, Why}+ and assumes you have died - but you are really alive.
{\sl Do not abuse.}
\item exit reason \verb+kill+ is unstoppable.
\end{itemize}

\subsection{Crashing and exit propagation}

\includegraphics[width=10cm]{images/exits.jpg}

\includegraphics[width=10cm]{images/crashing.jpg}

\begin{itemize}
\item Exits propagate through linked processes. 
\item If any of A..F crash all processes crash
\item System processes stop the propagation of errors. 
\item Process D  does not crash
\end{itemize}

\subsection{Why do we do this?}

We can build the system in layers to make it fault tolerant.

\includegraphics[width=10cm]{images/layers.jpg}



\subsection{Things to think about}

\begin{itemize}
\item If you do \verb+spawn+ followed by \verb+link+ the process might
  die very quickly (ie, before getting to the \verb+link+ statement)
\item Solution: \verb+spawn_link+ is like \verb+spawn+ followed by \verb+link+ only the two are performed atomically
\item Need to make sure \verb+trap_exit+ is evaluated *before* we try
  to catch errors -- think about synchronization here
\end{itemize}


\subsection{Exercise - Keep Alive}

Write a function \verb+make_global(Name, Fun/0)+ that starts a
"global" process named \verb+Name+.  If \verb+Name+ dies for any
reason, restart it. Record the times and reasons for starting and
restarting in an error log.

Useful stuff:

\begin{itemize}
\item \verb+time() -> {Hour, Minute, Second}+
\item \verb+error_logger:format(String, List)+ writes to the error log
\end{itemize}

\subsection{The principle of remote error handling}
\begin{itemize}
\item To make a fault-tolerant system you need two (separated) computers.
\item Let one process do the work
\item Let some other process fix the error
\item You can think of exit signals as uncaught exceptions that escape from the
process and propagate to some other process
\item Processes that fail should die-early
\item We can make very reliable systems this way
\item We can test  on one node and deploy on multiple nodes
(everything works the same way)
\end{itemize}

\subsection{half links (monitor)}
\begin{itemize}
\item \verb+erlang:monitor(process, Pid)+
\item asymmetric
\item countable (if a process is monitored N times it must be demonitored N times
before the monitor is released - unlike link/unlink)
\end{itemize}

\subsection{Summary}
\begin{itemize}
\item \verb+link+ - create a link
\item \verb+unlink+ - release the link
\item \verb+process_flag(trap_exit, true)+ -- trap exits
\item \verb+{'EXIT',Pid,Why}+ then message sent when a process dies
\end{itemize}

All of this works in distributed Erlang. These mechanism are orthogonal to
the \verb+spawn+, \verb+send+ and \verb+receive+

Now we've seen all that you need to make a powerful fault-tolerant system.

\section{Sockets Based Distribution}

\subsection{Uses gen\_tcp or gen\_udp}

Book \verb+chapter 14+

\subsection{Client}

\begin{verbatim}
{ok, Socket} = gen_tcp:connect(Host, Port, [Options]),
ok = gen_tcp:send(Socket, Data),
...
receive
    {tcp, Socket, Data} ->
         %% do something with the data
         ...
    {tcp_closed, Socket} ->
         %% take care of this ...    
\end{verbatim}

\subsection{Server}

A "single-shot" server (accepts one connection)

\begin{verbatim}
start_server() ->
    {ok, Listen} = gen_tcp:listen(Port, [Options]),
    {ok, Socket} = gen_tcp:accept(Listen),
    %% We use same calls as the client to read and write
    %% the socket
\end{verbatim}

\subsection{A sequential server}

\begin{verbatim}
{ok, Listen} = gen_tcp:listen(Port, ...)
seq_loop(Listen).

seq_loop(Listen) ->
    {ok, Socket} = gen_tcp:accept(Listen),
    loop(Socket),
    seq_loop(Listen).
\end{verbatim}

\subsection{A parallel server}

Now turn this into a parallel server

\begin{verbatim}
{ok, Listen} = gen_tcp:listen(Port, ...)
spawn(fun() -> par_loop(Listen) end).

par_loop(Listen) ->
    {ok, Socket} = gen_tcp:accept(Listen),
    spawn(fun() -> par_loop(Listen) end),
    loop(Socket).
\end{verbatim}

\subsection{Packet lengths}

A 4 byte length header is automatically added/removed by the system
when calling \verb+gen_tcp:send+ and messages are assembled to the
correct length before \verb+{tcp, Sock, Data}+ messages are send to
the controlling process

\begin{verbatim}
gen_tcp:connect(Host, Port, [..., {packet,4}, ...])
gen_tcp:listen(Port, [..., {packet, 4}, ...])
\end{verbatim}


\subsection{Sending Erlang terms}

\begin{verbatim}
gen_tcp:send(Socket, term_to_binary(Term)) ...

receive
        {tcp, Socket, Data} ->
              Term = binary_to_term(Data),
              ...
\end{verbatim}

\subsection{The middle man pattern}

\begin{verbatim}
loop(Pid, Socket) ->
    receive
        {Pid, Msg} ->
            gen_tcp:send(Socket, term_to_binary(Msg)),
            loop(Pid, Socket);
            
        {tcp, Socket, Data} ->
            Pid ! binary_to_term(Data),
            loop(Pid, Socket);
            
        {'EXIT', Pid} ->
            gen_tcp:close(Socket);
            
        {tcp_closed, Socket} ->
            exit(Pid, socket_closed)
    end.
\end{verbatim}

\subsection{Exercise}

Do this at home!

\verb+multi_server+ over sockets

Write \verb+server.erl+ and \verb+client.erl+.

Use a middle man to connect a client on one machine to a server on
another machine. Use \verb+middle_man.erl+ to connect the client to a
socket on one machine and on the other machine to connect a process
owning a socket to an instance of \verb+multi_server+.

Hint: Book page 248 (and these notes) have the parallel server
pattern.  The middle man pattern is described in the course notes and
on page 404 of the book in the description of \verb+lib_chan_mm+

\subsection{The Bit Syntax}

Pack/Match bit strings from binaries

\begin{verbatim}
1> Red = 2, Green=61, Blue=20.
20
%% Red takes 5 bits, Green 6, and blue 5
2> Bin = <<Red:5, Green:6, Blue:5>>.
<<23,180>>
3> io:format("~8.2.0B ~8.2.0b~n", binary_to_list(Bin)).
   00010111 10110100
%% 00010 111101 10100
\end{verbatim}

Size must be a multiple of 8 bits

\subsection{Tips}

\begin{itemize}
\item Can handle different endian words, big, little, ...
\item complex syntax
\item experiment in shell then cut and paste into program
\end{itemize}

\subsection{Examples}

\begin{itemize}
\item Book page 86 - find MP3 headers (for syncing with SHOUTCAST server)

\item Too numerous to mention

\item Fun just now. Decode/Encode AMF (Actionscript binary protocol)

\item Useful for writing assemblers, protocol convertors etc.
\end{itemize}

\section{ Distributed Erlang}

\begin{itemize}
\item Distributed Erlang -- several \verb+nodes+ belong to the same
  system.  \verb+spawn+ has an extra argument
  \verb+spawn(Node, Mod, Fun, Args)+
\item SMP Erlang -- symmetric multiprocessing. Two or more identical
  processors are connected to a single shared main memory.
\item Socket Distribution -- Not really distributed at all
\end{itemize}

\subsection{Distribution primitives}

Adds three new primitives and a load of libraries

\begin{itemize}
\item \verb+spawn(Node, Mod, Func, Args)+ -- {\sl everything} works
  as before regarding \verb+links+ exit messages an so on.
\item \verb+alive(Node)+ -- tell the system you are alive
\end{itemize}

The main libraries

\begin{itemize}
\item \verb+rcp+ -- doing remote procedure calls
\item \verb+global+ -- global operation over all nodes
\end{itemize}

\subsection{Distributed Erlang}

\begin{itemize}
\item Cookie based security
\item Distribution over TCP (can be over secure sockets, but
  installation is more complex)
\item Only suitable in a cluster
\item Used in enterprise software behind a firewall
\end{itemize}

\subsection{Distributed Erlang }

The most common architectural pattern is to write traffic handling as
non distributed applications on a single node. We use \verb+mnesia+
and the error loggers running replicated on multiple nodes.

\verb+mnmesia+ the (Erlang database) is can be configured to provide
table replication over multiple nodes.

Fault tolerant applications usually use replicated pairs of nodes. One
node does the work the other is a hot standby.

\subsection{Two nodes, one machine}

\begin{verbatim}
$ erl -sname one          %%  in one terminal window
(one@joe-armstrongs-computer)1> 
\end{verbatim}


\begin{verbatim}
$ erl -sname two        %% in another window
> (two@joe-armstrongs-computer)1> net_adm:ping('two@joe-armstrongs-computer').
pong
> (two@joe-armstrongs-computer)1> net_adm:ping('one@joe-armstrongs-computer').
pong
(two@joe-armstrongs-computer)2> node().
'two@joe-armstrongs-computer'
(two@joe-armstrongs-computer)3> rpc:call('one@joe-armstrongs-computer', 
                                         erlang, node, []).
'one@joe-armstrongs-computer'
\end{verbatim}


\subsection{Shell can connect to remote nodes}

\begin{verbatim}
$erl -name one
..
(one@joe-armstrongs-computer.local)1> ^G
User switch command
 --> j
   1* {shell,start,[init]}
 --> r 'server+doris.myerl.home.net'
 --> j
   1  {shell,start,[init]}
   2* {'server@doris.myerl.home.net',shell,start,[]}
 --> c 2
Eshell V5.5.5  (abort with ^G)
(server+doris.myerl.home.net)1> node().
'server@doris.myerl.home.net' ^G
User switch command
 --> c 1
(one+joe-armstrongs-computer.local)1> node().
'one@joe-armstrongs-computer.local'
\end{verbatim}

\section{OTP}

\includegraphics[width=10cm]{images/OTP_Arch.jpg}

\begin{itemize}
\item Open Telecomms Platform
\item Maintained by product group inside Ericsson
\item Releases 2-3 times/year
\item "http://www.erlang.org/":http://www.erlang.org/ 
\item "http://www.erlang.org/doc.html":http://www.erlang.org/doc.html
\end{itemize}

\subsection{Structure}

Generic structure

\begin{verbatim}
/appname/ebin  %% beam code
        /src   %% erlang source
        /priv  %% everything else
\end{verbatim}

\subsection{Principles}

\begin{itemize}
\item "http://www.erlang.org/doc/design\_principles/part\_frame.html":http://www.erlang.org/doc/design\_principles/part\_frame.html
\item Overview
\item Client-server \verb+gen_server+
\item Finite State machines \verb+gen_fsm+
\item Event handling \verb+gen_event+
\item Supervisor \verb+gen_sup+
\item Applications 
\item Releases
\item Application upgrade
\end{itemize}

\subsection{Getting started with applications}

\begin{itemize}
\item Read book \verb+chapters 16 and 18+ {\sl especially} +16.1
\item Read PhD thesis "http://www.sics.se/~joe/thesis":http://www.sics.se/~joe/thesis
\item Read {\sl design principles} 
   "http://www.erlang.org/doc/design\_principles/part\_frame.html"
\item Forthcoming O'Reilly book 
\end{itemize}

\subsection{behaviors}

\begin{itemize}
\item The OTP name for "design patterns"
\item Callback modules for client-servers, supervision trees etc.
\item Encapsulates "best practice" from many individual projects
\item 3'rd generation - ie the third rewrite of basic servers
\item Used in practice
\end{itemize}

\subsection{Case studies}
See Armstrong PhD thesis.
\begin{itemize}
\item AXD301 -- gen\_server (122), gen\_event (36), supervisor (20), gen\_fsm
(10), application (6).  
\item Nortel Networks -- gen\_server (56),
supervisor (19), application (15), gen\_event (9), rpc\_server (2),
gen\_fsm (1), supervisor\_bridge (1).
\end{itemize}

\verb+gen_server+ is the most used behavior

\subsection{gen\_server}

\begin{itemize}
\item A generic client-server model
\item \verb+gen_server:start_link(Name, Mod, InitArgs, Opts)+
\item \verb+Mod:handle_call(...)+ gets called for all RPCs
\item \verb+Mod:handle_cast(...)+ gets called for all casts
\item ...
\item \verb+Mod:code_change(...)+ gets called when you want to change the code
\end{itemize}

More later...

\subsection{gen\_supervisor}

\includegraphics[width=10cm]{images/sup.jpg}

\begin{itemize}
\item Supervision trees
\item Hierarchical tree of workers and supervisors
\item Supervisors monitor the workers
\item If a worker fails this is noticed by the supervisor
\item Supervisors can start, restart and kill workers
\item 1:1 and 1:N supervision
\end{itemize}

\section{Windup}

\subsection{Sequential Erlang}

\begin{itemize}
\item Simple functional language
\item Lists, tuples, atoms, bignums, floats, ...
\item Function selection is by pattern matching
\item Data selection is by pattern matching
\item Variables are immutable
\end{itemize}

\subsection{Concurrent Erlang}

\begin{itemize}
\item Adds \verb+spawn+, \verb+send+ and \verb+receive+ to sequential Erlang.
\item \verb+register+ [\verb+unregister+] can be used to associate a name with a process
\end{itemize}

\subsection{Fault-tolerant Erlang}

\begin{itemize}
\item \verb+catch+ .. \verb+throw+ and \verb+try+ ... \verb+catch+ ... \verb+end+ added to sequential Erlang
\item \verb+link+, \verb+process_flag(trap_exit, true)+ added to concurrent Erlang
\end{itemize}

\subsection{Distributed Erlang}
\begin{itemize}
\item Add +\verb+spawn(Node, Mod, Func, Args)+ to concurrent Erlang
\item  Or use explicit term passing over sockets
\end{itemize}

\subsection{Benefits of Erlang}
\begin{itemize}
\item Multi-core ready
\item Processes in the language (not OS)
\item Designed for fault-tolerant distributed programming
\item Battle tested
\end{itemize}

\subsection{Some Projects to research}
\begin{itemize}
\item AXD301 (Ericsson)
\item Kreditor (kreditor.se)
\item SimpleDB (Amazon)
\item CouchDB (text db)
\item MociWeb (Mochimedia)
\item Ejabberd (jabber server)
\item ErlyWeb (Erlang on Rails :-)
\end{itemize}

\subsection{And Remember...}

Make It Fun!

\end{document}

